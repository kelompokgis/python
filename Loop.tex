
\section {Python Loops}

Dalam sebuah artikel oleh Berkah Santoso yang menyatakan bahwa pada Python, kita dapat menggunakan statement for dan while. Pada statement while,  biasanya memiliki ciri berupa pengecekan kondisi dan perulangan dilakukan diawal. Sedangkan pada statement for, memiliki ciri berupa inisialisasi perulangan dilakukan diawal statement dan perulangan tersebut akan berhenti ketika syarat atau kondisi yang telah ditentukan terpenuhi. \cite{santoso2009bahasa}

Secara umum, pernyataan pada bahasa pemrograman akan dieksekusi secara berurutan. Pernyataan pertama dalam sebuah fungsi dijalankan pertama, diikuti oleh yang kedua, dan seterusnya. Tetapi akan ada situasi dimana Anda harus menulis banyak kode, dimana kode tersebut sangat banyak. Jika dilakukan secara manual maka Anda hanya akan membuang-buang tenaga dengan menulis beratus-ratus bahkan beribu-ribu kode. Untuk itu Anda perlu menggunakan pengulangan di dalam bahasa pemrograman Python. 
 


Dalam bahasa pemrograman Python pengulangan dibagi menjadi 3 jenis, yaitu : 
\begin{enumerate}
\item
While Loop 
\item
For Loop 
\item
Nested Loop 
\end{enumerate}




\section {While Loop} 
Pengulangan While Loop di dalam bahasa pemrograman Python dieksesusi statement berkali-kali selama kondisi tersebut benar atau True. 
Berikut adalah contoh penggunaan pengulangan While Loop.
 

\begin {verbatim}
count = 0 
while (count < 9): 
        print ('The count is:', count) 
        count = count + 1 
print ("Good bye!") 
\end{verbatim} 




 
\section {For Loop} 
Pengulangan For pada Python memiliki kemampuan untuk mengulangi item dari urutan apapun, seperti   list atau string. 
Berikut adalah contoh penggunaan pengulangan For Loop. 
\begin {verbatim}
angka = [1,2,3,4,5] 
for x in angka: 
        print(x)
		
		
		
Nested Loop 
Bahasa pemrograman Python memungkinkan penggunaan satu lingkaran di dalam loop lain. Bagian berikut menunjukkan beberapa contoh untuk menggambarkan konsep tersebut.   
Berikut adalah contoh penggunaan Nested Loop. 


\begin{verbatim}
i = 2 
while(i < 100): 
        j = 2 
        while(j <= (i/j)): 
                if not(i   \%  j): break 
                j = j + 1 
        if (j > i/j) : print i, " is prime" 
        i = i + 1 
print "Good bye!" 
\end{verbatim}
Perhatikan contoh berikut ini:
\begin{verbatim}
print ("1") 
print ("2") 
print ("3") 
print ("4") 
print ("5") 
print ("6") 
print ("7") 
print ("8") 
print ("9") 
print ("10") 
\end{verbatim}

Contoh koding diatas merupakan contoh program untuk menampilkan angka 1 sampai dengan 10 tanpa perulangan. Tanpa menggunakan perulangan, programmer harus menuliskan semua statement secara manual sehingga source code menjadi lebih panjang.
Dengan menggunakan perulangan, source code lebih pendek dan efisien. 
Berikut contoh program untuk mencetak angka 1 sampai dengan 10 dengan menggunakan perulangan.

\begin{verbatim}
i = 1 
while(i < 11): 
~~~ print(i) 
~~~ i = i+1 
\end{verbatim}

Selain membahas 3 jenis perulangan diatas, tutorial ini membahas control perulangan, yaitu: 
Break Statement 
Continue Statement 
dan Pass Statement 


\subsection{FOR Loop} 
FOR Loop digunakan untuk melakukan perulangan atau iterasi sampai batas atau range yang telah ditentukan.
Dibawah ini adalah sintak dasar FOR Loop di Python.

\begin{verbatim} 
for iterating     var in range: 
~~ statements(s) 
\end{verbatim}

Fungsi   range()   biasanya digunakan sebagai counter pada perulangan bentuk For. range(10) artinya menampikan perulangan sebanyak 10 elemen.
Apabila program diatas Anda jalankan, maka akan menampilkan angka 1 sampai dengan 10 seperti output di bawah ini:

 
1 
2 
3 
4 
5 
6 
7 
8 
9 
10 
 
 Program mencetak angka -1 s/d 8 
\begin{verbatim}
i = 10 
for i in range(-10, 10, 2):       range(range awal, range akhir, selisih) 
~~ print(i) 
\end{verbatim}

Perhatikan pada range(-10, 10, 2) artinya perulangan akan dimulai dari batas awal -10 sampai dengan batas akhir 10 dengan selisih 2.

Apabila program diatas Anda jalankan, maka akan menampilkan output berikut ini:

 
-10 
-8 
-6 
-4 
-2 
0 
2 
4 
6 
8 

Contoh 3

 
Program menampilkan huruf Belajar Python 
for~huruf~in 'Belajar Python':    
~~ print (huruf) 

Apabila program diatas Anda jalankan, maka akan menghasilkan output berikut ini:

 
B 
e 
l 
a 
j 
a 
r 
  
P 
y 
t 
h 
o 
n 

Contoh 4

Program berikut akan menampilkan perulangan dari list atau tupple.

 
Program menampilkan huruf Belajar Python 

makanan = ['Pizza', 'Nasi Bebek',~ 'Rujak Buah'] 
for makan in makanan: 
~~ print ("Makanan Favorit :", makan) 

Apabila program diatas Anda jalankan, maka akan menghasilkan output berikut ini:

 
Makanan Favorit : Pizza 
Makanan Favorit : Nasi Bebek 
Makanan Favorit : Rujak Buah 



While Loop 
While Loop akan menjalankan statemet selama kondisi terpenuhi (atau bernilai true).

Di bawah ini adalah sintak dasar dari While Loop pada Python

Contoh Program

Coba Anda ketik program di bawah ini:

 
Program mencetak angka 1 s/d 10 

i = 1 
while(i < 11): 
 print(i) 
 i = i+1 

Apabila program diatas Anda jalankan, maka akan menghasilkan output seperti di bawah ini:

 
1 
2 
3 
4 
5 
6 
7 
8 
9 
10 

Perintah while pada python merupakan perintah yang paling umum digunakan untuk proses
iterasi. Konsep sederhana dari perintah while adalah ia akan mengulang mengeksekusi statemen dalam
blok while selama nilai kondisinya benar. Dan ia akan keluar atau tidak melakukan eksekusi blok
statemen jika nilai kondisinya salah.
 Bentuk umum statemen while,
 while (kondisi) :
 statemen
 Contoh penggunaan while :
contoh 1 : 
\begin{verbatim}
>>> while True :
 ... print "Tekan CTRL + C untuk Stop"
 ...
end{verbatim}

Pada contoh 1, merupakan contoh sederhana penggunaan while. Pada contoh di atas
program akan terus mengeksekusi statemen dalam badan while, dikarenakan kondisinya selalu
benar (true). Kondisi seperti ini disebut infinite loop.i
contoh 2 : 
\begin{verbatim}
>>> x = "Gunadarma"
 >>> while x:
 ... print x, ' '
 ... x = x[1:]
 ...
 Gunadarma
 unadarma
 nadarma
 adarma
 darma
 arma
 rma
 ma
 a
 end{verbatim}
 
contoh 3 : 
\begin{verbatim}
>>> a = 0; b = 10
 >>> while a < b :
 ... print a,
 ... a = a + 1
 ...
 0 1 2 3 4 5 6 7 8 9
 end{verbatim}

Perintah for dalam python mempunyai ciri khas tersendiri dibandingkan dengan bahasa
pemrograman lain. Tidak hanya mengulang bilangan-bilangan sebuah ekspresi aritmatik, atau
memberikan keleluasaan dalam mendefinisikan iterasi perulangan dan menghentikan perulangan pada
saat kondisi tertentu. Dalam python, statemen for bekerja mengulang berbagai macam tipe data
sekuensial seperti List, String, dan Tuple.
 Bentuk umum perintah for,
 for (variabel) in (objek) :
 statemen
 else:
 statemen 
 
 contoh 4:
 \begin{verbatim}
 >>> for i in [5, 4, 3, 2, 1]:
 ... print i,
 ...
 5 4 3 2 1 
 end{verbatim}
 
Pada contoh 1, perintah perulangan terjadi dimana data-data untuk iterasi (objek) berada
dalam List. Jadi elemen-elemen yang berada dalam List akan di masukkan (assign) ke dalam variabel
target yaitu i. 

contoh 5
\begin{verbatim}
>>> T = [(1,2), (3,4), (5,6)]
 >>> for (a,b) in T :
 ... print (a,b)
 ...
 (1, 2)
 (3, 4)
 (5, 6) 
 end{verbatim}
 
Pada contoh 2, merupakan penggunaan tipe data Tuple untuk proses perulangan. Elemen
pada tuple akan di assign kedalam variabel a dan b. 

Perintah break digunakan untuk menghentikan jalannya proses iterasi pada statemen for
atau while. Statemen yang berada di bawah break tidak akan di eksekusi dan program akan keluar dari
proses looping. 

contoh break

\begin{verbatim}
>>> x = 1
 >>> while x < 5:
 ... if x == 3:
 ... break
 ... print x
 ... x = x+1 
 ... else:
 print "Loop sdh selesai dikrjkn"
 ...
 1
 2 
 end{verbatim}
 
 perintah continue
Statemen continue menyebabkan alur program kembali ke perintah looping. Jadi jika
dalam sebuah perulangan terdapat statemen continue, maka program akan kembali ke perintah looping
untuk iterasi selanjutnya. 
contoh continue

\begin{verbatim}
>>> n = 10
 >>> while n:
 ... n = n - 1
 ... if n % 2 != 0:
 ... continue
 ... print n
 ...
 8
 6
 4
 2 
 end{verbatim}
 
perintah pass
Statemen pass mengakibatkan program tidak melakukan tindakan apa-apa. Perintah pass
biasanya digunakan untuk mengabaikan suatu blok statemen perulangan, pengkondisian, class, dan
fungsi yang belum didefinisikan badan programnya agar tidak terjadi error ketika proses compilasi. 

contoh pass
\begin{verbatim}
#program tidak akan melakukan
 #proses looping
 while True : pass 
end{verbatim}

Perulangan while
Perulangan while akan mengulang didalam ruang lingkup while, selama suatu kondisi terpenuhi. 

contoh perulangan while
\begin{verbatim}
>>> n = 9
>>> while n < 20 :
... print n
... n = n + 1
9
10
11
12
13
14
15
16
17
18
19
end{verbatim}

Pada contoh diatas, nilai variabel n akan ditambahkan 1 secara terus menerus sampai kondisi
n lebih kecil dari 20. 

Fungsi range()
Jika Anda ingin melakukan perulangan sejumlah yang diinginkan, fungsi built-in range sangat
membantu. Fungsi tersebut menghasilkan sejumlah indeks dari nilai yang telah ditentukan.
contohnya

\begin{verbatim}
>>> range(15)
[0, 1, 2, 3, 4, 5, 6, 7, 8, 9, 10, 11, 12, 13, 14]
end{verbatim}
Ataupun sebagian angka yang diinginkan.

contohnya
\begin{verbatim}
>>> range (8, 15)
[8, 9, 10, 11, 12, 13, 14]
>>> range(0,9,3)
[0, 3, 6]
>>> range(0, 20, 3)
[0, 3, 6, 9, 12, 15, 18] 
end{verbatim}
Contoh diatas menunjukan kelipatan dari suatu interval bilangan yang mempunyai sintaks
range(<nilai-awal>, <nilai-akhir>, <kelipatan-angka>). 

Perintah break, continue dan else
Perintah break seperti dalam bahasa C, berarti keluar dari ruang lingkup yang terkecil dari
kondisi for atau while.
Perintah continue sama halnya dengan di C, yang berfungsi melanjutkan kalimat perintah
berikutnya dalam kondisi perulangan.
Pada kondisi perulangan juga diperbolehkan untuk menggunakan kalimat perintah else, yang
dijalankan pada saat kondisi perulangan for tidak menemui suatu kondisi atau jika suatu kondisi tersebut
mengalami kesalahan / false (dengan while), tetapi bukan pada saat kondisi perulangan dihentikan
dengan perintah break.
contoh

\begin{verbatim}
for n in range(2, 10):
 for x in range(2, n):
 if n % x == 0:
 print n, 'sama dengan', x, '*', n/x break
else:
 print n, 'adalah bilangan prima'
2 adalah bilangan prima
3 adalah bilangan prima 4 sama dengan 2 * 2
5 adalah bilangan prima 6 sama dengan 2 * 3
7 adalah bilangan prima 8 sama dengan 2 * 4
9 sama dengan 3 * 3
end{verbatim}

Penjelasannya adalah apabila suatu kondisi dalam perulangan for x in range(2, n) tidak ada yang
terpenuhi maka alur perulangannya akan lari ke ruang lingkup perintah else. 
 
Infinite Loop 

Infinite Loop adalah kondisi perulangan, dimana statement akan dijalankan terus menerus tanpa berhenti. Akan berhenti kalau Anda menekan tombol CTRL+C.

Di bawah ini contoh program Infinite Loop

 
program menampilkan tulisan Python tanpa henti 

flag = 1 

while (flag): print ("Python") 
print ("Good bye!") 



Nested Loop 

Nested Loop secara sederhana adalah perulangan di dalam perulangan.

Di bawah ini adalah sintak dasar Nested Loop pada Python:

 
for iterating     var in sequence: 
~~ for iterating     var in sequence: 
~~~~~ statements(s) 
~~ statements(s) 

atau yang menggunakan while loop

 
while expression: 
~~ while expression: 
~~~~~ statement(s) 
~~ statement(s) 

Contoh Program

Di bawah ini adalah contoh program implementasi Nested Loop untuk mencetak bilangan prima dari 2 sampai 30.

 
Program menampilkan bilangan prima dari 2 s/d 30 

i = 2 
while(i < 30): 
~~ j = 2 
~~ while(j <= (i/j)): 
~~~~~ if not(i   \%  j): break 
~~~~~ j = j + 1 
~~ if (j > i/j) : print (i, " adalah bilangan prima") 
~~ i = i + 1 

print ("Good bye!") 


Apabila program diatas Anda jalankan, maka akan menampilkan output seperti di bawah ini.

 
2~ adalah bilangan prima 
3~ adalah bilangan prima 
5~ adalah bilangan prima 
7~ adalah bilangan prima 
11~ adalah bilangan prima 
13~ adalah bilangan prima 
17~ adalah bilangan prima 
19~ adalah bilangan prima 
23~ adalah bilangan prima 
29~ adalah bilangan prima 

Pengulangan adalah salah satu hal penting yang ada di bahasa pemrograman. Pengulangan digunakan misalnya untuk meng-update   nama   file   yang cukup banyak jumlahnya, atau mengakses piksel satu persatu pada gambar. 
Python memiliki tiga jenis pengulangan yang wajib Anda cermati untuk membuat sebuah aplikasi dengan Python. Pengulangan yang pertama adalah   while. Dengan menggunakan   while, Anda dapat membuat kondisi tertentu untuk menghentikan   while. Biasanya   while   digunakan untuk melakukan   loopingyang tidak pasti. Coba lihat contoh berikut (Anda dapat menulisnya dalam sebuah   file, kemudian eksekusi   file   tersebut di konsol): 
i = 0 
while True: 
~~~ if i < 10: 
~~~~~~~ print "Saat ini i bernilai: ", i 
~~~~~~~ i = i + 1 
~~~ elif i >= 10: 
~~~~~~~ break 

Pada potongan kode diatas,   while   akan terus berputar selama i masih kurang dari 10. Jika sudah lebih dari 10 maka   while   akan berhenti. Pengulangan   whilejuga biasa digunakan di aplikasi konsol, untuk menahan   user   mengisikan semua input yang diperlukan dan baru akan berhenti setelah semua input dan proses interaksi berakhir. Jika kode diatas kita jalankan, maka   output-nya akan seperti ini: 

Saat~ini i bernilai:  0 
Saat~ini i bernilai:  1 
Saat~ini i bernilai:  2 
Saat~ini i bernilai:  3 
Saat~ini i bernilai:  4 
Saat~ini i bernilai:  5 
Saat~ini i bernilai:  6 
Saat~ini i bernilai:  7 
Saat~ini i bernilai:  8 
Saat~ini i bernilai:  9 

Sekarang kita coba gunakan   for. Pengulangan   for   biasa digunakan untuk pengulangan yang sudah jelas banyaknya. Misal, Anda ingin mengulang sebuah pengulangan sampai 10 kali atau mengeluarkan semua hasil   query   dari   databasedi halaman HTML. Berikut ini adalah contoh kode untuk pengulangan   for: 
for i in range(0, 10): 
~~~ print i 
Jika dijalankan maka kode diatas akan mengeluarkan   output   seperti ini: 

0 
1 
2 
3 
4 
5 
6 
7 
8 
9 
Tidak hanya mengiterasi deretan angka, pengulangan   for   pun dapat Anda gunakan untuk mengulang sesuatu yang   iterable   seperti   list,   tuple,   dictionary, dan   iterable object   lainnya. Berikut ini kita ambil contoh dengan mengulang sebuah   list   yang berisi karakter anime Dragonball Super: 

dragonball     super     character = ["Son Goku", "Vegeta", "Beerus", "Trunks", "Whiz", "Champa"] 
for character in dragonball     super     character: 
~~~ print character 

Jika kita jalankan potongan kode tadi, maka   output-nya akan seperti berikut: 

Son Goku 
Vegeta 
Beerus 
Trunks 
Whiz 
Champa 
For Loop 
Seperti pada bahasa pemrograman lainnya, for loop sudah menjadi standar namun berbeda-beda tata cara penulisan nya di setiap pemrograman. 

Sekarang kita langsung buat contoh di Python.    


Contoh iterasi pada String  

for~n in 'Python':   
~~~ print 'Huruf :', n 

  
iterasi pada List biasa 

mobil = ['sedan', 'truk', 'angkot']  
for p in mobil: 
~~~ print 'Mobil :', mobil 


iterasi pada list melalui index 
for i in range(len(mobil)): 
~~~ print 'Mobil :', mobil[i] 

iterasi angka / range 

for a in range(1,10): 
~~~~ print "Angka :", a 
~~~~ if(a == 5):      ditambah conditional 
~~~~~~~~ print "Saya dapat angka : ",a 

iterasi loop nested 
for a in range(1,10): 
~~~ for x in range(11,20): 
~~~~~~~~b~=~a~* x      
~~~~~~~ print "Angka :", b 

loop dgn break 
for letter in 'Python': 
~~ if letter == 'h': 
~~~~~ break 
~~ print 'Current Letter :', letter 

print "Good job !!!" 

While Loop 
WHile dipakai untuk looping dimana iterasi akan dilakukan selama kondisi yang diberikan benar. While ini juga bisa di pakai untuk Infinite loop. 

Contoh While 
count = 0 
while count < 100: 
           print "Count ke : ", count 
           count = count + 1 

infinite loop 
''' 
Set loop ini untuk kondisi dimana suatu syarat tidak pernah TRUE 
''' 

setvar =1 
while setvar == 1 
        input = input     raw("Masukan angka :") 
        print "Angka anda : ", input 

loop diatas akan berhenti jika anda stop manual misal dgn CTRL+C di terminal 
''' 
ELSE statement di while loop. di Python kita bisa set WHile loop lalu dikasih kondisi 
''' 
count = 0 
while count < 5: 
           print "count : ",count 
           count = count + 1 
else: 
        print "Lihat yang masuk sini apa : ",count 

while dgn break 
angka~=~10~~~~~~    
while~angka~>~0:~~~~~~~~~~     
~~  
~~ print 'Angka :', angka 
~~ angka = angka -1 
~~ if angka == 7: 
~~~~~ break 




\section {implementasi}
Kami telah menerapkan eksekusi ini dalam simbolis inti loop-extended komponen yang dijelaskan sebelumnya di OCaml, dan format protokolnya. 
Keterkaitannya di OCaml dikombinasikan dengan kode C dan Python untuk diintegrasikan dengan parser off-the-shelf. 
Kami menggunakan biner yang ada analisis infrastruktur [4,42] untuk mengambil jejak eksekusi dan mendapatkan semantik instruksi x86.

\section {For}
Ini adalah (mungkin) perulangan yang paling sering dipakai dalam C++. Sintaknya adalah sebagai berikut:
for (inisialisasi; kondisi; iterasi) yang ingin dilakukan...
            Dalam for, anda bisa melakukan inisialisasi, yaitu mendeklarasikan variabel baru dan langsung memasukan nilai di dalamnya, kemudian memberi kondisi dimana perulangan itu akan terus dilakukan, dan pada akhirnya menentukan iterasi. Contoh dari for adalah sebagai berikut:

#include 
using namespace std;

void main () {
int num_masuk;
cout << “berapa banyak anda mau melakukan perulangan?: “;
cin >> num_masuk;

for (int i = 1; i <= num_masuk; ++i) {
cout << “ini adalah baris ke-“ << i << “\n” ;

            Pada kode di atas, yang akan menjadi input dari user untuk menentukan berapa banyak perulangan akan dilakukan adalah num_masuk, dan variabel yang akan dijadikan patokan dalam perulangan adalah i, di mana i di buat di dalam perintah for dan kemudian langsung di assign (inisialisasi).
            Kondisi yang digunakan adalah kondisi dimana i <= num_masuk yang berarti, jika / selama i lebih kecil atau sama dengan num_masuk maka perulangan akan terus dilakukan. Iteasi yang ditentukan di sini adalah nilai dari i akan ditambah 1 (satu) dalam setiap perulangan (++i).


Contoh program yang sudah di compile dari kode ini adalah:
Namun for juga bisa digunakan sebagai berikut:
for (int i = 1; i <= 10; ++i) {
for (int j = 1; j <= 10; ++j) {
cout << i * j << “\n”;

/section {Evaluasi Program}
Sebagai studi kasus dengan skala penuh, kami mengambil 3 contoh dunia nyata yang termasuk Windows dan program Linux yang diketahui memiliki kerentanan buffer overflow. 
Ini termasuk program yang ditargetkan oleh Slammer yang terkenal worm pada tahun 2003, yang terpengaruh oleh kerentanan GDI baru-baru ini pada tahun 2007, dan server HTTP. 
Untuk ghttpd, alat kita menemukan dua kerentanan buffer overflow di fungsi Log di util.c. 
Salah satunya dijelaskan pada penelitian terdahulu menggunakan program subjek ini. Overflow baru melibatkan yang terpisah penyangga dan butuh perbaikan terpisah. 

/section {WHILE And WEND}
Proses loop dengan statement WHILE ... WEND juga akan membentuk sebuah blok loop. Blok WHILE ... WEND sama dengan blok DO WHILE ... LOOP.

Bentuk dasarnya adalah :
   WHILE ekspresi_kondisi
      'baris kode yang di-loop
   WEND

Contohnya adalah prosedur bernama Contoh1_DoWhileLoop di atas dapat ditulis dalam blok WHILE ... WEND menjadi :
   Public Sub Contoh1_WhileWend()
      Dim lNomor As Long, lBaris As Long
    
      'hapus yang lama dulu ya
      Sheets("Dataku").Columns(13).ClearContents
    
      'mulai proses contoh
      lNomor = 3
      While lNomor <= 8
         lBaris = lNomor + 2
         Sheets("Dataku").Range("M" & lBaris).Value = lNomor
         lNomor = lNomor + 1
      Wend
   End Sub
